\documentclass[a4paper]{report}

\usepackage[utf8]{inputenc} % un package
\usepackage[T1]{fontenc}      % un second package
\usepackage[french]{babel}  % un troisi�me package
\usepackage{fancyhdr}
\usepackage{graphicx} %images 
\usepackage{layout}
\newcommand{\up}[1]{\textsuperscript{#1}}

\headsep = 42pt
\pagestyle{fancy}
\fancyhf{}

\lhead{ \includegraphics[scale=0.38]{logo-esilv_rem.jpg}\\
\small ASSAOUI Ramez - GREGORI Anthony}
\rhead{ \includegraphics[scale=0.8]{logo_acensi_rema.jpg}\\
\small STOJANOVIC Maxime - HILALI Elies}

\begin{document}      
	\smallskip\begin{center}
	\large\textbf{ Reunion du 06/10/2016}
	\hrule
	\end {center}
	
	
	\underline{\large \textbf{Ordre du jour :}}
	\smallskip
	\begin{itemize}
	\item[--] Analyse : quel est le probleme, les sous problemes ?  
	\item[--] Planning : qui, quand, o\'u et pourquoi ?  
	\item[--] Deadlines : 11 octobre, 8 novembre
	\end{itemize}
	
	\medskip\underline{\large \textbf{R\'eunion :}}
	\smallskip	
	\begin{itemize}
	\item[] Points abord\'es : 
	\begin{itemize}
	\newline
	\item[--] Organisation : \\ 
	Nous sommes vu comme 4 chefs de projet technique, chapot\'e par Florianne et Fabrice avec une aide de Beno\^it ou de Jean pour l'infrastructure. 
	\\Chaque semaine  une r\'eunion sera faites, pas forc\'ement avec tous les membres de l'\'equipe, ces r\'eunions se feront majoritairement avec Florianne. \\
	Nous allons utiliser l'outil jira : https://fr.atlassian.com/software/jira avec nos 4 boites mails : \newline anthony.gregori78200@gmail.com \newline elies.hilali@devinci.fr \newline ramez.aissaoui@hotmail.com \newline  maxime.stojanovic@gmail.com \newline et notre adresse mail commune est : projetpi2.esilv@gmail.com. L'outil GIT est li\'e \'a jira. 
	\newline
	\item[--] Planning : \\ 
	1 mois d'\'etude : Quels sont les acteurs, payant/gratuit, les technos utilis\'ees, le prix, comment peut-on faire la m�me chose. Nous devons donc \'etudier la concurrence et d\'efinir la/les meilleurs outils, infra (exemple : IBM, Google). L'objectif est de trouver entre 5 et 10 solutions en d\'etaillant l'infra, les supports, les co\^uts macros. 
	\\Ensuite, 2 mois sur tests/POC et \'ecriture  du projet. Avec d\'efinition du co\^uts exact, un planning plus pr\'ecis. 
	\newline
	\item[--] Projet : \\ 
	La plus grande probl\'ematique que nous allons avoir est selon Fabrice la v\'elocit\'e, au niveau du volume de donn\'ees cela ne devrait pas d\'passer, pour le moment, 100Go. Nous allons avoir \'a terme, \'a gerer les API,les logs applicatifs, des donn\'ees en clair, et les donn\'ees les moins propres que nous pouvons avoir sera du CSV. L'objectif sera de monitorer l'activit\'e des sites internet. Nous \'egalement toutes les donn\'ees non g\'er\'e par KOIOS, donc des donn\'ees plus grosses, mais moins sensible et moins critique. Nous aurons aussi certainement \'a g\'erer des logs (1000/s). 
	\end{itemize}
	\end{itemize}
	
	
	%ici le texte 
\medskip

%\underline{\large \textbf{A faire pour la prochaine fois}}
	\smallskip \newline 
	
	La prochaine r\'eunion se fera le mardi 18 octobre avec Benoit th\'eoriquement, dans la matin\'e. 
%ici le texte 	\\
			
\end{document}